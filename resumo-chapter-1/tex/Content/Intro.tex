\section{Introdução}
\par
Sistemas de Recomendações(SR) são ferramentas de software e técnicas que fornecem sugestões de itens que sejam de uso ao usuário. As sugestões tem relação com diversos processos de decisão, como quais itens comprar, quais músicas ouvir ou o que ler.
\par
``Item'' é o termo geral usado para denotar o que o SR recomenda aos usuários. Um SR normalmente foca em um tipo específico de item e de acordo com seu design, sua interface gráfica e a técnica de recomendação base usada para gerar sugestões úteis e eficazes para aquele tipo de item.
\par
Em sua forma mais simples, recomendações personalizadas são oferecidas como listas ranqueadas de itens. Ao fazer esse ranking, os SR tentam prever qual produto ou serviço mais adequados, baseado nas preferências do usuário. Para completar essa tarefa computacional, os SR coletam dos usuários suas preferências, o que são explícitamente expresso, ou inferidos por interpretar suas ações.
\par
Apesas do estudo de SR ser relativamente novo comparado a pesquisas em outros campos de sistemas de informação clássicos e técnmicas (ex.: bandos de dados ou mecanismos de pesquisa). SR surgiu como uma área independente de pesquisa no meio da década de 1990. Nos anos recentes o interesse em sistemas de recomendação tem aumentado drasticamente, e alguns fatos mostram o mesmo:
\begin{enumerate}

\item Sistemas de recomendação fazem um papel importante em diversos serviços de internet bem classificados como Amazon.com,YouTube,Netflix,Yahoo e Tripadvisor. Dessa forma muitas companhias de conteúdo estão desenvolvendo e SR como parte dos serviços que elas proveem a seus assinantes. Um grande exemplo foi o prêmio de um milhão de dóllares que o Netflix ofereceu ao primeiro time que conseguisse melhorar substancialmente o desempenho do seu sistema de recomendações.

\item Existem conferências e workshops dedicados para o campo de estudo. Especificamente falando, a ACM Recommender Systems (RecSys), estabelecida em 2007 e agora é o mais importante evento anual em tecnologia de pesquisa em se aplicação de SR.

\item O assunto tem se tornado cadeira em alguns cursos de Educação Superior no mundo para graduados e não graduados. Tutoriais de SR são muito populares em conferencias de Ciência da computação e recentemente um livro introduzindo técnicas de SR foi publicado.

\item Vários jornais acadêmicos possuem áreas dedicadas para publicação de papers de SR, como o AI Communications (2008); IEEE Intelligent Systems (2007); Inter-
national Journal of Electronic Commerce (2006); International Journal of Computer Science and Applications (2006); ACM Transactions on Computer-Human Interaction (2005); and ACM Transactions on Information Systems (2004).

\end{enumerate}
