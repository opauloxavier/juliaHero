\section{Resumo}

\subsection{Definição}
\par
\textit{Collaborative Filtering} é um algoritmo de recomendação popular que baseia suas predições e recomendações nas avaliações ou comportamento de outros usuários no sistema. A principal suposição por trás desse método é que as opiniões de outros uários podem ser selecionadas e agregadas de maneira que é possivel ter uma predição razoável das preferências um usuário alvo. Intuitivamente, assume-se também, que se os usuários concordam sobre a qualidade ou relevância de alguns itens, então eles também vão provavelmente concordar sobre outros itens.
\par
Dentro de \textit{Collaborative Filtering}, existem alguns conceitos que são necessários para descrever o domínio do problema e alguns requerimentos particulares de cada sistema. Muitos desses conceitos são também compartilhados por outros métodos de recomendação. São exemplos:

\begin{itemize}

  \item Domínio de Informação
  \item Rating,User, Item
  \item Rating Matrix
  \item Predict

\end{itemize}

\par
A grande maioria dos algoritmos de CF usados atualmente, operam inicialmente gerando predições de preferência usuário e então produizindo as recomendações deles ranqueando os itens candidatos por preferências previstas.

\subsection{Tipos de Algoritmos de Filtragem Colaborativa}

\par
Algumas técnicas de recomendação por CF sofreram avanços recentes devido ao anteriormente citado \textit{``Prêmio Netflix''} e serão brevemente citadas nesse resumo:

\subsubsection{Modelos de Fatorização de Matrizes}

\par
Modelos de Fatorização de Matrizes mapeam ambos usuários e itens em junção com um espaço articular latente de dimensão \textit{f},assim como interaçãoes usuário-item são modeladas como produtos internos nesse espaço. O espaço latente tenta explicar as avaliações por caracterização de ambos os fatores de porodutos e usuários automaticamente inferidos pelo feedback do usuário. Esses fatores podem ser de dimensões óbvias, como gêneros de filmes, ou dimensões menos bem definidas, como filmes ``mais adultos'', ou elementos completamente não interpretáveis.

\subsubsection{Modelos de Vizinhança}
\par
É a abordagem mais comum para CF's. Em sua forma original, é compartilhado com virutalmente todos os CF iniciais é baseado em usuário-usuário. De maneira geral, essa abordagem possui muita popularidade, devido as seguintes características:

\begin{itemize}

  \item \textbf{Simplicidade:} São intuitivos e relativamente simples de implementar. Em sua forma mais simples, apenas um parametro (número de vizinhos usados na predição) necessita de ajuste.
  \item \textbf{Justificabilidade:} Fornecem justificativas concisas e intuitivas para as predições computadas. Para alguns sistemas é de extrema importância que se explique as recomendações ao usuário.
  \item \textbf{Eficiência:} Um dos pontos fortes dos métodos de vizinhança é sua eficiência. Diferentemente dos sistemas baseados em modelos, eles não precisam de fases de treinamento custosas, o que precisaria ser feito em intervalos frequentes em aplicações comerciais grandes.
   \item \textbf{Estabilidade:} Outra propriedade útil nesse tipo de abordagem é que ele é pouco afetado pela adição constante de usuários, itens e avaliações.
\end{itemize}